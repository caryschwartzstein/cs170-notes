\chapter{Shortest Paths}
Define \(\operatorname{dist}[u,v]\) as the length of the shortest path from \(u\) to \(v\).

\section{Problem}
Given \(G = \left(V, E\right)\) and source node \(s\), compute shortest paths from \(s\) to every node in \(G\).

\section{Simple case}
Pretend first that all edges are the same length.
Then all you have to do is breadth-first search, as the BFS tree explores nodes one layer at a time.

\begin{algorithm}
	\caption{Use a queue for BFS.}
	\begin{algorithmic}
		\Function{QBFS}{$G, s$}
			\State \(Q\leftarrow\) empty queue
			\State \(\text{visited}\leftarrow 1\) at \(s\), 0 everywhere else
			\State \(\text{enqueue}(Q, s)\)
			\While{not empty \(Q\)}
				\State \(x\leftarrow \operatorname{dequeue}(Q)\)
				\State \(\text{visited}[x]\leftarrow 1\)
				\For{neighbor \(x'\) of \(x\)}
					\If{not \(\operatorname{visited}[x']\)}
						\State \(\text{enqueue}(Q, x')\)
					\EndIf
				\EndFor
			\EndWhile
		\EndFunction
	\end{algorithmic}
\end{algorithm}


\begin{algorithm}
	\caption{BFS counting distance.}
	\begin{algorithmic}
		\Function{QBFS}{$G, s$}
			\State \(\text{dist}[s] \leftarrow 0\)
			\State \(Q\leftarrow\) empty queue
			\State \(\text{visited}\leftarrow 1\) at \(s\), 0 everywhere else
			\State \(\text{enqueue}(Q, s)\)
			\While{not empty \(Q\)}
				\State \(x\leftarrow \operatorname{dequeue}(Q)\)
				\State \(\text{visited}[x]\leftarrow 1\)
				\For{neighbor \(x'\) of \(x\)}
					\If{not \(\operatorname{visited}[x']\)}
						\State \(\text{enqueue}(Q, x')\)
						\State \(\text{dist}[x'] \leftarrow \text{dist}[x] + 1\)
					\EndIf
				\EndFor
			\EndWhile
		\EndFunction
	\end{algorithmic}
\end{algorithm}

BFS and DFS are distinguished precisely by their choice of data structure.
If in the above algorithm, you used a stack instead of a queue,
the algorithm would perform DFS.

\section{Harder case}
Instead of all edges' having the same length use \(\ell: \left(\text{edges}\right)\to \mathbb{R}^+\).
We are doomed! A dumb solution would be to add tons of degree-two vertices
along long edges.

\subsection{Intuition}
Take our undirected graph as a road network.
Suppose that there is an oil spill/zombie virus epidemic that is centered at our
particular source node.
Note, then, that the oil is spreading in every way at once, at a uniform
\(1\) unit per minute.
In this way the oil is certain to achieve the shortest path.
We may identify the \emph{time} at which the oil reaches a node with the
\emph{length} of the shortest path to the same node.

\begin{quote}
	\em
	(Simon) this is the first time I encountered the floodfill
	characterization of Dijkstra's algorithm! It is a very helpful illustration,
	much easier to grasp than dynamic-programming node costs.
\end{quote}

In order to implement a chronology of events, we need an data structure with
the following operations:
\begin{description}
	\item[insert] a vertex to schedule it for a time
	\item[delete min] return and forget the next event
	\item[decrease key] change the time of a scheduled event
\end{description}

\begin{tabular}{lll}
	Operation & Binary heap & \(d\)-ary heap \\\hline
	insert & \(\log n\) & \(\log_d n\) \\
	delete min & \(\log n\) & \(d \log_d n\)\\
	decrease key & \(\log n\) & \(\log_d n\)
\end{tabular}

\subsection{Dijkstra's algorithm}
\begin{algorithm}
	\caption{Compute shortest paths.}
	\label{dijkstra}
	\begin{algorithmic}
		\Function{DijkSP}{$G, s$}
			\State \(\text{dist}[v] \leftarrow \infty\) for all \(v \in G\)
			\State \(Q\leftarrow\) empty priority queue (any implementation)
			\State \(\text{insert}(Q, (v, \infty))\) for all \(v \in G\)
			\State \(\text{dist}[s] \leftarrow 0\)
			\State \(\text{decreasekey}(Q, (s, 0))\)
			\While{not empty \(Q\)}
				\State \((u, t)\leftarrow \text{deletemin}(Q)\)
				\State \(\text{dist}[u] \leftarrow t\)
				\For {edge \(u\to v\)}
					\If {\(\text{dist}[v] > \text{dist}[u] + \ell(u, v)\)}
						\State \(\text{decreasekey}(Q, (v, \text{dist}[u] + \ell(u, v))\)
					\EndIf
				\EndFor
			\EndWhile
		\EndFunction
	\end{algorithmic}
\end{algorithm}

In Algorithm~\ref{dijkstra}, there are \(\left|V\right|\) \textsc{insert}s and \textsc{deletemin}s and \(\left|E\right|\) \textsc{decreasekey}s.
Each priority queue operation takes log time, so altogether Dijkstra's
completes in \(O\left(\left(\left|V\right| + \left|E\right|\right)\log \left|V\right|\right)\).

\subsection{Proof of correctness}
\begin{enumerate}
	\item Distance values keep decreasing. Nodes later in your path are getting closer to your destination!
	\item All edge lengths are positive. Otherwise, it could be tricky.
	\item Let \(R_l = \left\{\text{\(l\)th closest vertices to \(s\)}\right\}\).
	\item The next step at time \(l\) is drawn from \(R_{l + 1} - R_{l}\).
\end{enumerate}

Proof by induction on the step \(l\).
\begin{description}
	\item[induction hypothesis] All the distances in \(R_l\) are computed correctly.
	\item[base case] \(l = 0\)
	\item[induction] Let \(v\) be the \((l + 1)\)th closest vertex. It is correctly accounted for in the \textsc{descreasekey} invocation.
\end{description}
