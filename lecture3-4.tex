\chapter{Fast Fourier Transform}
\section{Complex numbers}
\emph{alright we all know this let's move on\,\ldots}let \(\mathbb{C} = \mathbb{R}[\xi]/\left(\xi^2+1\right)\)\ldots%
use a chart \(\mathbb{R}^2\to\mathbb{C}, \left(x,y\right)\mapsto xe^{iy}\).

\subsection{\(n\)th roots of unity}
Divide the complex unit circle \(\left\|x\right\| = 1\) into \(n\) equally spaced points including 1.
These are the \(n\)th roots of unity, and there are \(n\) of them.

On the complex unit circle, they are equally spaced points from \(1\).
If \(\omega = \exp\frac{2\pi i}{n}\), the roots of unity are \(\left\{1, \omega, \ldots, \omega^{n - 1}\right\}\).

\chapter{FFT cont.}
\section{Polynomial evaluation/interpolation}
Given \(p(x) = p_0 + \ldots + p_{n - 1}x^{n - 1}\) and \(\alpha \in \mathbb{C}\), compute \(p(\alpha)\).
A na\"ive algorithm accomplishes this in \(\Theta(n)\)\footnote{This fact is surprisingly hard to understand: remember that you can compute all of the powers efficiently by remembering each previous power.}

\section{Fourier transform}
Given a polynomial \(p(x) = p_0 + \ldots + p_{n - 1}x^{n - 1}\), evaluate \(p(\omega^i)\) for all \(0 \leq i < n\). A na\"ive algorithm takes \(\Theta(n^2)\) time (the proof of this fact is left as an exercise to the reader).

\section{Fast Fourier transform}
\subsection{More properties of \(n\)th roots of unity}
If \(n = 2^m\), then you can get all of them by \(x \mapsto \pm\sqrt{\cdot}\), which makes you a binary tree. The leaves at depth \(m\) (the root, 1, has depth 0) are the \(2^m\)th roots of unity.

\begin{itemize}
	\item \(n\)th roots appear in \(+/-\) pairs.
	\item Squaring \(n\)th roots of unity results in \(\frac{n}{2}\)th roots of unity.
\end{itemize}

If \(p(\omega) = 0 + 1\omega + \ldots + 7\omega^7\), then \(p(-\omega) = \left(0 + 2\omega^2 + 4\omega^4 + 6 \omega^6\right) - \left(1\omega + 3\omega^3 + 5\omega^5 + 7\omega7\right)\) (even terms minus odd terms).

Let \(p_e\) contain the even terms and \(p_o\) the odd terms.
\begin{align}
	\begin{pmatrix}
		p(\omega^0)\\
		\vdots\\
		p(\omega^3)\\
		p(\omega^4)\\
		\vdots\\
		p(\omega^8)
	\end{pmatrix}
	&=
	\begin{pmatrix}
		p_e(\omega^0) + p_o(\omega^0) \\
		\vdots \\
		p_e(\omega^3) + p_o(\omega^3) \\
		p_e(\omega^0) - p_o(\omega^0) \\
		\vdots \\
		p_e(\omega^3) - p_o(\omega^3)
	\end{pmatrix}
\end{align}

The FFT will split an \(n\)-FT to 2 \(\frac{n}{2}\)-FTs with \(O(n)\) root work for a combined runtime of \(\Theta(n\log n)\).

If \(p_e\) is degree 6 in \(x\), it is degree 3 in \(t\). Then let \(P(x) = A_e\left(x^2\right) + xA_o\left(x^2\right)\).

\begin{align}
\begin{pmatrix}
p(\omega^0)\\
\vdots\\
p(\omega^3)\\
p(\omega^4)\\
\vdots\\
p(\omega^8)
\end{pmatrix}
&=
\begin{pmatrix}
A_e\left(\left(\omega^0\right)^2\right) + A_o\left(\left(\omega^0\right)^2\right) \\
\vdots \\
A_e\left(\left(\omega^3\right)^2\right) + A_o\left(\left(\omega^3\right)^2\right) \\
A_e\left(\left(\omega^0\right)^2\right) - A_o\left(\left(\omega^0\right)^2\right) \\
\vdots \\
A_e\left(\left(\omega^3\right)^2\right) - A_o\left(\left(\omega^3\right)^2\right) \\
\end{pmatrix}
\end{align}